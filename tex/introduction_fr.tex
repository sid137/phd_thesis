\chapter*{Introduction}
\addstarredchapter{R\'esum\'e en Francais}
\markboth{Introduction}{}
\label{introduction_fr} 


\subsection*{L'information quantique et l'information classique}



Le domaine de l'Information Quantique s'est rapidement d\'evelopp\'e durant les derni\`eres d\'ecennies. La m\'ecanique quantique a ouvert une nouvelle voie permettant de manipuler lumi\`ere et mati\`ere afin d'observer des ph\'enom\`enes sans \'equivalent classique. Certains principes centraux de la m\'ecanique quantique, tels que le principe d'incertitude de Heisenberg, ou celui de la superposition des \'etats, nous donnent la possibilit\'e d'observer ces ph\'enom\`enes. Ainsi est n\'ee l'id\'ee d'appliquer la m\'ecanique quantique \`a la th\'eorie de l'information. L'\'el\'ement le plus fondamental de l'information quantique est le bit quantique, ou \emph{qubit}. Alors que les bits classiques permettent de repr\'esenter l'information comme des valeurs discr\`etes 0 ou 1, les qubits, grace \`a la superposition quantique, permettent de repr\'esenter l'information comme 0, 1, ou une superposition de ces deux valeurs.

 
Cette propri\'et\'e du qubit a amen\'e au d\'eveloppement de nouveaux protocoles li\'es au transfert d'information, au calcul, et \`a l'informatique, qui se r\'ev\`eleraient impossible en consid\'erant des bits d'information classique. En 1981, Feynmann fit l'une des premi\`eres propositions pour b\'en\'eficier d'une application des propri\'et\'es quantiques \`a l'information, qui \'etait d'utiliser un ordinateur quantique afin de simuler l'\'evolution d'un syst\`eme quantique  \cite{feynman1982simulating}. Ensuite, en 1984, Bennet et Brassard ont d\'evelopp\'e un protocole de distribution de cl\'es cryptographiques en utilisant des canaux quantiques \cite{Bennett84}. Deutsch suivit peu apr\`es, en 1985, en proposant un mod\`ele de machine de turing quantique, ce qui a donn\'e le moyen d'analyser les algorithmes quantiques en utilisant des portes logiques quantiques \cite{Deutsch85}. Ekert a continu\'e l'exploration du transfert de l'information quantique, en d\'eveloppant un protocole en 1991 pour la communication quantique fond\'e sur l'intrication quantique \cite{Ekert91}. Des recherches concernant l'utilisation de l'information quantique pour le calcul ont continu\'e pendant les ann\'ees 1990 avec le d\'eveloppement de l'algorithme de Shor en 1994, qui montre comment factoriser de grands nombres en utilisant un ordinateur quantique \cite{Shor94}, puis avec l'algorithme de Grover, qui permet d'utiliser l'information quantique pour chercher un \'el\'ement dans une base de donn\'ees non-tri\'ees \cite{Grover96}.

 
Tous ces protocoles concernant la manipulation de l'information quantique requi\`erent la conservation de l'\'etat de superposition quantique des bits quantiques. La n\'ecessit\'e de pr\'eserver cette superposition nous oblige \`a \'eviter de mesurer la valeur d'un qubit, ce qui le forcerait \`a prendre une valeur bien d\'efinie, et donc reviendrait \`a d\'etruire ses propri\'et\'es quantiques. Ceci devient probl\'ematique si nous essayons de mettre en oeuvre ces protocoles avec nos conceptions classiques de l'information, car la m\'ecanique quantique impose le th\'eor\`eme de non-clonage, qui interdit de cr\'eer des copies exactes d'\'etats quantiques inconnus. Cette limitation a motiv\'ee la recherche de nouveaux moyens de pr\'eserver l'information quantique durant la manipulation et le stockage.

 

\subsection*{M\'emoires Quantiques}

Un \'el\'ement cl\'e pour la mise en oeuvre de beaucoup de protocoles d'information quantique est une m\'emoire quantique r\'eversible, qui permettrait de stocker et r\'ecup\'erer des \'etats quantiques.  Une utilisation d'une m\'emoire quantique, par exemple, serait comme source de photons uniques sur demande, ce qui servirait comme \'el\'ement important dans le calcul quantique. Une m\'emoire quantique pourrait \'egalement r\'esoudre le probl\`eme critique du transfert de l'information quantique \`a longue distance. Les photons se propageant dans une fibre optique sont soumis aux pertes, ce qui limite la distance sur laquelle il est possible de transf\'erer des qubits optiques. Le r\'ep\'eteur quantique pourrait alors \^etre une solution, en utilisant l'intrication entre deux photons situ\'es deux extr\'emit\'es du canal de communication, ce qui n'est possible que si nous utilisons une m\'emoire quantique pour stocker nos \'etats quantiques. C'est ce contexte qui a motiv\'e notre group \`a travailler vers le d\'eveloppement d'une m\'emoire quantique.



\subsection*{Recherche au Laboratoire Kastler Brossel}

Pendant les vingt derni\`eres ann\'ees, l'\'equipe d'Optique Quantique du Laboratoire Kastler Brossel s'est concentr\'ee sur l'\'etude des effets quantiques de l'interaction lumi\`ere-mati\`ere dans les atomes de C\'esium. Plusieurs exp\'eriences ont \'et\'e lanc\'ees pour \'etudier la r\'eduction du bruit quantique en cavit\'e ou avec des atomes froids, par Laurent Hilico \cite{HilicoPhD}, Astrid Lambrecht \cite{LambrechtPhD}, Thomas Coudreau \cite{CoudreauPhD}, et Vincent Josse \cite{JossePhD}.  Durant leurs th\`eses, Laurent Vernac \cite{VernacPhD} et Aurelian Dantan \cite{DantanPhD} ont d\'evelopp\'e des mod\`eles th\'eoriques concernant les fluctuations \'electromagn\'etiques quantiques et leur transfert vers des atomes via l'interaction lumi\`ere-mati\`ere. Plus r\'ecemment, les travaux de Jean Cviklinksi \cite{CviklinskiPhD} et Jememie Ortalo  \cite{ortalo} ont men\'e au d\'eveloppement et \`a la caract\'erisation d'une m\'emoire pour les \'etats coh\'erents et bas\'ee sur des atomes chauds, ainsi qu'une \'etude exp\'erimentale de la transparence induite \'electromagn\'etiquement dans le C\'esium.

 

 

\newpage
\section*{Plan de la th\`ese}

Cette th\`ese d\'ebute en partie \ref{part:1} avec une partie th\'eorique. Elle aborde des principes g\'en\'eraux d'optique quantique utiles au travail exp\'erimental r\'ealis\'e. Nous introduisons tout d'abord les \'etats quantiques du champ \'electromagn\'etique, ainsi que la repr\'esentation de ces \'etats grace \`a la matrice de densit\'e et \`a la fonction de Wigner. Ensuite, nous discutons des propri\'et\'es optiques d'un faisceau traversant un mat\'eriau non lin\'eaire, et de la possibilit\'e d'utiliser ces interactions pour la g\'en\'eration des \'etats comprim\'es.

Dans la partie \ref{part:2}, nous examinons le montage exp\'erimental utilis\'e pour cr\'eer un oscillateur param\'etrique optique, qui permet de g\'en\'erer des \'etats du vide comprim\'es r\'esonants avec la raie D2 du C\'esium \`a 852 nm. Puis, nous abordons les techniques de tomographie quantique et de maximale vraisemblance, permettant de caract\'eriser ces \'etats. Nous terminons avec une discussion des m\'ethodes explor\'ees pour transformer notre source continue de vide comprim\'e en impulsions compatibles avec notre m\'emoire quantique.

Dans la troisi\`eme et derni\`ere partie, partie \ref{part:3}, nous retra\c{c}ons le d\'eveloppement d'une nouvelle exp\'erience qui nous permettra d'utiliser les atomes froids de C\'esium, pi\'eg\'es dans un MOT, comme milieu de stockage. Le montage de cette exp\'erience requiert un assortiment de nouveaux outils et de techniques exp\'erimentales, dont nous discutons la mise en place. Nous voyons la fa\c{c}on dont ils nous aident \`a progresser vers le stockage des \'etats quantiques dans un ensemble d'atomes de C\'esium, et finalement vers l'intrication de deux ensembles atomiques.

\chapter*{Conclusion}
\addstarredchapter{Conclusion}
\markboth{Conclusion}{}
\label{conclusion} 

The recent development of the field of Quantum Information Processing has lead to the discovery of new protocols for using quantum information in communication and computational applications.  A quantum memory for light would serve as a core component in many of these applications such as quantum computing and long distance quantum information transfer, as it would allow us to synchronize the processing of quantum states.  The technique of storing states of light in an atomic medium via EIT is currently a promising area of research due to its potential of obtaining reliable information transfer with long memory storage times.  Our work during this thesis has focused on the novel goal of developing a quantum memory for squeezed light in a cold Cesium vapor.

In the first part of this thesis, we carried out a theoretical  overview of Nonlinear and Quantum Optics in order to understand the mechanisms behind the generation of squeezed vacuum states.  Next, we examined the development of a doubly degenerate OPO operating below threshold, which we used to generate -3 dB of squeezed vacuum resonant with the Cesium $D_2$ line at 852 nm.  We then carried out a Quantum State Tomography of the homodyne measurements of these states, and by using the maximum likelihood estimation method, were able to reconstruct the state's density matrix and Wigner function.  Finally we examined two methods of interfacing our squeezed vacuum with the Cesium atoms by creating squeezed pulses of light.

Once we obtained pulsed source of squeezed light, we began the construction of a quantum memory for Cesium.  We examined the characteristics of a MOT that was built to cool and trap Cesium atoms, and we discussed the experimental techniques developed to create lasers at the necessary transitions, control the MOT's magnetic field, and time and synchronize different aspects of the experiment.  We also discussed the results of characterizing the MOT performance with respect to its optical density, and our ability to cut the magnetic field.  

\section*{Current Status and Future Outlook}

The development of our quantum memory in cold Cesium atoms currently continues with the further optimization of the MOT performance, and the usage of Raman spectroscopy to completely cancel the magnetic field.  Once these milestones are attained, work will turn towards the measurement of EIT in the MOT, and the storage and retrieval of a coherent state in our memory in order to establish a baseline memory efficiency.  The successful storage of a coherent state will then allow us to continue with the storage of a squeezed state of light, and eventually the storage of entanglement in two separate quantum memories.  Once these steps have been achieved, this will show the potential of using a Cesium based quantum memory as an element of quantum information processing.


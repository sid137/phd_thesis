\chapter*{Foreword}
\addstarredchapter{Foreword}
\markboth{Foreword}{}
\label{Foreword} 

Until recently, our knowledge gained by observing the universe rested on a simple foundation -- the idea of certainty; the certainty of repeated causes yielding repeated effects, the certainty of 1 vs. 0, or something vs. nothing.  This seemingly simple concept has directed our observations of everything that we define to be the \emph{classical} world, where \emph{classical information} follows our expectations of certainty.

The development and advance of quantum mechanics however, has destroyed this illusory foundation of certainty.  On the contrary, we know believe that the universe is governed by a fundamentally uncertain, and probabilistic set of laws -- those of quantum mechanics.  As we can establish the idea of defining information that obeys our classical expectations as classical information, we can extend our vocabulary by defining quantum information as that which is subject to the laws and uncertainties of quantum mechanics.  As a consequence of this transition, we are now forced to reassess how the uncertainty of information affects everything built on our prior classical foundations.

\subsection*{Quantum Information vs. Classical Information}
 
The principles of quantum mechanics render quantum information susceptible to phenomenon which have no classical parallel.  We can only represent an unmeasured quantum system as existing in a superposition of states, which opens the possibility of any real measurement of identically prepared systems to produce several different outcomes.  Upon measurement, the Heisenberg Uncertainty Principle renders it impossible to precisely observe all possible aspects of a quantum system with a simultaneously, thus limiting our attainable knowledge of the system.  Additionally, we must consider the existence of the multiple possible outcomes creating quantum interference with each other, and thus further affecting the end result of our measurements.  Imprecise knowledge of a system further limits how we can manipulate it -- the no-cloning theorem prevents us from making an identical copy of an unknown quantum state, thus preventing us from storing and distributing quantum information in the same way as classical information.  All of these phenomenon have prompted a wide array of research in the field of quantum optics to explore ways of properly manipulating, and benefiting from the non-classical properties of quantum information.


\subsection*{Historical and Global Context}


Research Groups
who else is building a quantum memory (1 pg)

Research Approaches (0.5 - 1)
how are they doing it  (0.5-1 pg)





\subsection*{Research at the Laboratoire Kastler-Brossel}

Over the last 20 years, the Continuous Variable Quantum Optics group at the Laboratoire Kastler-Brossel has focused on studying the quantum-optical effects of light-matter interactions in Cesium atoms.  There have been a variety of experiments carried out to study the quantum noise reduction in cavities and with cold atoms by Laurent Hilico \cite{HilicoPhD}, Astrid Lambrecht \cite{LambrechtPhD}, Thomas Coudreau \cite{CoudreauPhD}, and Vincent Josse \cite{JossePhD}.  The theses of Laurent Vernac \cite{VernacPhD} and Aurelien Dantan \cite{DantanPhD} have developed the theoretical work concerning quantum electromagnetic fluctuations, and their transfer towards atoms via light-matter interactions.  Most recently, the work of Jean Cviklinski \cite{CviklinskiPhD} and Jeremie Ortalo \cite{ortalo} has yielded the development and characterisation of an atomic memory for coherent states with warm atoms, and an experimental study of electromagnetically-induced transparency in Cesium.

 

\newpage
\section*{Thesis Outline}

Part \ref{part:1} of this thesis begins with an overview of the general quantum optic concepts used to carry out the experimental work shown here.  We define the quantum states of the electromagnetic field, and show how we can represent those states using the density matrix and Wigner function.  We then proceed to discuss the nonlinear interactions of light passing through a nonlinear material, and show how we can use these interactions to generate squeezed states.

In Part \ref{part:2}, we look at the experimental setup used to create an optical parametric oscillator, which allows us to generate squeezed vacuum states resonant with the Cesium D2 line at 852 nm.  We then look at the techniques used to characterize these states using quantum homodyne tomography and iterative maximum likelihood estimation.  We finish by discussing the approaches developed to convert our continuous source of squeezed light into pulses compatible with our quantum memory.

Finally in Part \ref{part:3}, we look at the development of a new experiment which would allow us to use cold Cesium atoms in our recently developed magneto-optical trap as a storage medium for our quantum memory.  As this depends on an array of novel tools and experimental techniques, we will discuss the development of these elements which has furthered our progress towards storing quantum states onto our Cesium atoms, and eventually entangling two atomic ensembles.
